\documentclass[10pt,twocolumn,letterpaper]{article}
\usepackage{cvpr}
\usepackage{times}
\usepackage{graphicx}
\usepackage{amsmath}
\usepackage{amssymb}
\usepackage[pagebackref=true,breaklinks=true,colorlinks,bookmarks=false]{hyperref}
\begin{document}
\title{311-Service-Request Data Analysis using Apache Spark - SOEN 691 Project}
\author{Apoorv Semwal \and Hareesh Kavumkulath \and Loveshant Grewal}
\maketitle

\begin{abstract}
Recent advances in the field of Big Data Analytics and Machine Learning have introduced a plethora of open-source tools and technologies for both, 
academia and the massive data analyst community. In this project we try leveraging one such popular distributed data processing framework Apache Spark
to analyse 311 - Service Request Data for the city of XXXXX. Being updated almost on a daily basis for the last XXXXX years, large volume of this dataset makes it a suitable candidate for analysis using a distributed processing framework like Spark. Making use of Spark Ecosystem libraries like Spark SQL and Spark ML, on this dataset, enables us to derive some interesting insights for better planning of city's resource. Identifying 3 primary goals for this project we first try answering a few statistical questions like “*most frequent complaints reported*”, “*Average time to resolve the complaint/issue*”.
This involves making extensive use of Spark SQLs Dataframe API. Secondly we develop a predictive model for XXXXX, by evaluating performance of a set of 
selected supervised learning algorithms available in Spark ML. As part of our last goal we would be applying K-Means clustering over a selected set 
of features dividing the dataset into clusters and analysing them to identify any underlying service request patterns.
\end{abstract}

\section{Introduction}

\section{Section 2}

\textbf{1. Sample 1:} Its some random text:
\begin{itemize}
  \item Sample Item 1 \textbf{{\em Sample Italics}} in the dataset.
  \item Sample Item 1 \textbf{{\em Italics}}, 
\end{itemize}

\textbf{2. Sample 2:}\\
\begin{tabular}{ |p{3cm}|p{5cm}|  }
 \hline
 \multicolumn{2}{|c|}{Sample Table} \\
 \hline
 Sample1 &Sample\_1, Sample\_2, Sample\_3\_1\\
 \hline
 Sample1 &Sample\_1, Sample\_2, Sample\_3\_1\\
 \hline
 Sample1 &Sample\_1, Sample\_2, Sample\_3\_1\\
 \hline
\end{tabular}\\

\begin{figure}
   \begin{center}
  \includegraphics[width=\linewidth]{ConfusionMatrix.JPG}
   \end{center}
   \caption{Sample Confusion Matrix.\cite{W1}
   \label{Img1}}
\end{figure} 
Sample Reference of a figure in text. Figure~\ref{Img1}.\\

\begin{tabular}{ |p{0.5cm}||p{1.4cm}|p{1.4cm}|p{1.4cm}|p{1.4cm}|  }
 \hline
 \multicolumn{5}{|c|}{Multicol Table} \\
 \hline
 ID & Sample2 & Sample3 & Sample4 & Sample5\\
 \hline
 1 & 74.5 & 64.5 & 63.5 & 66.0\\
 \hline
 2 & 78.41 & 81.61 & 81.81 & 82.95\\
 \hline
\end{tabular}\\

\begin{equation}
\frac{\partial Sample Equation_{y_i}}{\partial X}
\end{equation}

\section{Conclusions}
\appendix
{\small
\bibliographystyle{cvpr_bibstyle}
\bibliography{bibliography}
}
\end{document}